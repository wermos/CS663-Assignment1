\documentclass[a4paper]{article}

\usepackage[DIV=9]{typearea}
\usepackage{microtype}
\usepackage{mathtools, amssymb}
\usepackage{bbm}

\renewcommand{\P}{\textsf{Prob}}

\title{3}
\date{}

\begin{document}
\maketitle
We are assuming that $I$ and $J$ are of same dimensions. Also, we assume that in $I+J$ corresponding pixel values of each image gets added directly. There is no after-scaling to compensate when the intensities go out of range.

Let $I$, $J$, $K$ be the random variables for these three images respectively. Now, for each location
\begin{equation}
	\begin{aligned}
		\P_K(k) &= \sum_{i=-\infty}^{\infty}\sum_{j=-\infty}^{\infty} \mathbbm{1}_{i+j=k} \cdot \P_{IJ}(i,j) &\\
		\P_K(k) &= \sum_{i=-\infty}^{\infty} \P_{IJ}(i,k-i) &(\text{since } \mathbbm{1}_{i+j=k} = 0 \text{ for} j\neq k-i)
	\end{aligned}
\end{equation}
Hence the PMF of image $I+J$ for each location is given by $\displaystyle\sum_{i=-\infty}^{\infty} \P_{IJ}(i,k-i)$.

Now, we can see that there is a presence of $i$ and $k-i$ akin to a convolution. In fact, we can make this idea more robust if both the distributions $p_I(i)$ and $p_J(j)$ were independent. Then, 
\begin{equation}
	\begin{aligned}
		\P_K(k) &= \sum_{i=-\infty}^{\infty} \P_{I}(i) \cdot \P_{J}(k-i) &\\
		\P_K(k) &= p_I(i) \ast p_J(j) &(\text{definition of convolution})\\
	\end{aligned}
\end{equation}
Hence the PMF of image $I+J$ for each location is given by $p_I(i) \ast p_J(j)$ when the distributions are independent. This is exactly the convolution operation as studied in class.
\end{document}
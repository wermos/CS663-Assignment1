\documentclass[a4paper,landscape]{article}

\usepackage[DIV=15]{typearea}
\usepackage{microtype}
\usepackage{enumitem}
\usepackage{mathtools, amssymb}
\usepackage{bbm}
\usepackage{nicematrix}
\setcounter{MaxMatrixCols}{20}
\setlength{\parindent}{0em}
\title{1}
\date{}

\begin{document}
\maketitle
Let the covolution mask be given by $w=[w(0); w(1); w(2); w(3); w(4); w(5); w(6)]$ and
the image vector be given by $f=[f(0);f(1);\ldots;f(n-1)]$.

Now, we will pad $\operatorname{len}(w)-1$ $(=6)$ zeroes on both side of the image to get the same answer as convolving $w$ with an $f$ of infinite size. Let this new vector be  \[f_p=[\underbrace{f(-6); f(-5);\ldots; f(-1)}_{0\text{'s}}; \underbrace{f(0); \ldots; f(n-1)}_{f}; \underbrace{f(n); \ldots; f(n+4); f(n+5)}_{0\text{'s}}]\]

Now, let $r=f \ast w$, then the equivalent truncated convolution $r_s=f_p \ast w$ where $r_s$  contains $n+6$ elements $[r_s(0); r_s(1); \ldots, r_s(n+5)]$ such that $r_s(i)=\displaystyle\sum_{j = 0}^{6} f(i-j)\cdot w_j $.

This can be written in a matrix form as shown below
\begin{equation}
    \underbrace{
    \begin{bmatrix}
        r_s(0)\\
        r_s(1)\\
        r_s(2)\\
        \vdots\\
        r_s(n+5)\\
    \end{bmatrix}
    }_{r_s} =
    \underbrace{
    \begin{bNiceMatrix}
        w(6)&w(5)&w(4)&w(3)&w(2)&w(1)&w(0)&0&\Cdots&&&0 &0&0&0&0&0&0\\
        0&w(6)&w(5)&w(4)&w(3)&w(2)&w(1)&w(0)&0&\Cdots&&0 &0&0&0&0&0&0\\
        0&0&w(6)&w(5)&w(4)&w(3)&w(2)&w(1)&w(0)&0&\Cdots&0 &0&0&0&0&0&0\\
        \vdots&\vdots&\vdots&\vdots&\vdots&\vdots&\vdots&\vdots&\vdots&\vdots&\vdots&\vdots&\vdots&\vdots&\vdots&\vdots&\vdots&\vdots\\
        0&0&0&0&0&0&0&\Cdots&0&w(6)&w(5)&w(4)&w(3)&w(2)&w(1)&w(0)&0&0\\
        0&0&0&0&0&0&0&\Cdots&&0&w(6)&w(5)&w(4)&w(3)&w(2)&w(1)&w(0)&0\\
        0&0&0&0&0&0&0&\Cdots&&&0&w(6)&w(5)&w(4)&w(3)&w(2)&w(1)&w(0)\\
        \CodeAfter
        \OverBrace[shorten,yshift=1mm]{1-1}{1-6}{f(-6); f(-5);\ldots; f(-1)}
        \OverBrace[shorten,yshift=1mm]{1-7}{1-12}{f(0); \ldots; f(n-1)}
        \OverBrace[shorten,yshift=1mm]{1-13}{1-18}{f(n); \ldots; f(n+5)}
    \end{bNiceMatrix}
    }_{M}
    \underbrace{
    \begin{bmatrix}
        f(-6)\\
        f(-5)\\
        f(-4)\\
        f(-3)\\
        f(-2)\\
        f(-1)\\
        f(0)\\
        f(1)\\
        \vdots\\
        f(n-1)\\
        f(n)\\
        f(n+1)\\
        f(n+2)\\
        f(n+3)\\
        f(n+4)\\
        f(n+5)\\
    \end{bmatrix}
    }_{f_p}
\end{equation}
This  $M$ is a $(n+6)\times (n+12)$ matrix. Moreover, we can remove first and last six columns of $M$ corresponding to multiplication with zero-padded elements of $f_p$. This gives us a $(n+6)\times n$ matrix $M_t$ with  same behaviour as $M$
\begin{equation}
    \underbrace{
    \begin{bmatrix}
        r_s(0)\\
        r_s(1)\\
        r_s(2)\\
        \vdots\\
        r_s(n+5)\\
    \end{bmatrix}
    }_{r_s} =
    \underbrace{
    \begin{bNiceMatrix}
        w(0)&0&\Cdots&&&&&&&&0\\
        w(1)&w(0)&0&\Cdots&&&&&&&0\\
        w(2)&w(1)&w(0)&0&\Cdots&&&&&&0\\
        w(3)&w(2)&w(1)&w(0)&0&\Cdots&&&&&0\\
        w(4)&w(3)&w(2)&w(1)&w(0)&0&\Cdots&&&&0\\
        w(5)&w(4)&w(3)&w(2)&w(1)&w(0)&0&\Cdots&&&0\\
        w(6)&w(5)&w(4)&w(3)&w(2)&w(1)&w(0)&0&\Cdots&&0\\
        0&w(6)&w(5)&w(4)&w(3)&w(2)&w(1)&w(0)&0&\Cdots&0\\
        \vdots&\vdots&\vdots&\vdots&\vdots&\vdots\\
        0&\Cdots&0&w(6)&w(5)&w(4)&w(3)&w(2)&w(1)&w(0)&0\\
        0&\Cdots&&0&w(6)&w(5)&w(4)&w(3)&w(2)&w(1)&w(0)\\
        0&\Cdots&&&0&w(6)&w(5)&w(4)&w(3)&w(2)&w(1)\\
        0&\Cdots&&&&0&w(6)&w(5)&w(4)&w(3)&w(2)\\
        0&\Cdots&&&&&0&w(6)&w(5)&w(4)&w(3)\\
        0&\Cdots&&&&&&0&w(6)&w(5)&w(4)\\
        0&\Cdots&&&&&&&0&w(6)&w(5)\\
        0&\Cdots&&&&&&&&0&w(6)\\
    \end{bNiceMatrix}
    }_{M_t}
    \underbrace{
    \begin{bmatrix}
        f(0)\\
        \vdots\\
        f(n-1)\\
    \end{bmatrix}
    }_{f}
\end{equation}
But, this pattern requires an additional condition that $n\geq 6$, and the matrix for $n<6$ case will be slightly different. Instead of making seperate cases, we can make size of $f$ greater than six by padding it with three zeroes in both  directions. Let this new vector be  \[f_s=[\underbrace{f(-3); f(-2); f(-1)}_{0\text{'s}}; \underbrace{f(0); \ldots; f(n-1)}_{f}; \underbrace{f(n);  f(n+1); f(n+2)}_{0\text{'s}}]\]

Also, we might truncate the resultant convolution vector to the size of $n$ by removing the first and last three entries and get $r_t$. So, $r_s$ can also be indexed using elements of this truncated $r_t$ as
\[r_s = r_t(-3); r_t(-2); r_t(-1); r_t(0); \ldots; r_t(n-1); r_t(n);  r_t(n+1); r_t(n+2)\]
This gives us a square matrix $M_s$ as follows given below
\section{Answer}
\begin{equation}
    \underbrace{
    \begin{bmatrix}
        r_t(-3)\\
        r_t(-2)\\
        r_t(-1)\\
        r_t(0)\\
        r_t(1)\\
        \vdots\\
        r_t(n-1)\\
        r_t(n)\\
        r_t(n+1)\\
        r_t(n+2)\\
    \end{bmatrix}
    }_{r_s} =
    \underbrace{
    \begin{bNiceMatrix}
        w(3)&w(2)&w(1)&w(0)&0&\Cdots&&&0 &0&0&0\\
        w(4)&w(3)&w(2)&w(1)&w(0)&0&\Cdots&&0 &0&0&0&\\
        w(5)&w(4)&w(3)&w(2)&w(1)&w(0)&0&\Cdots&0 &0&0&0\\
        \vdots&\vdots&\vdots&\vdots&\vdots&\vdots&\vdots&\vdots&\vdots&\vdots&\vdots&\vdots\\
        0&0&0&0&\Cdots&0&w(6)&w(5)&w(4)&w(3)&w(2)&w(1)\\
        0&0&0&0&\Cdots&&0&w(6)&w(5)&w(4)&w(3)&w(2)\\
        0&0&0&0&\Cdots&&&0&w(6)&w(5)&w(4)&w(3)\\
        \CodeAfter
        \OverBrace[shorten,yshift=1mm]{1-1}{1-3}{f(-3); f(-2); f(-1)}\
        \OverBrace[shorten,yshift=1mm]{1-4}{1-9}{f(0); \ldots; f(n-1)}
        \OverBrace[shorten,yshift=1mm]{1-10}{1-12}{f(n); f(n+1); f(n+2)}
    \end{bNiceMatrix}
    }_{M_s}
    \underbrace{
    \begin{bmatrix}
        f(-3)\\
        f(-2)\\
        f(-1)\\
        f(0)\\
        f(1)\\
        \vdots\\
        f(n-1)\\
        f(n)\\
        f(n+1)\\
        f(n+2)\\
    \end{bmatrix}
    }_{f_s}
\end{equation}
\section{Application}
Usually by doing convolution we lose some infomation of our input. But  with  this matrix representation we can calculate when do we lose the information and when the information is recoverable using the invertability of $M_s$. If $M_s$ is invertible, we can recover back $f_s$ from $r_s$ as follows else we can't recover $f_s$
\[f_s=M_s^{-1}r_s\]
\end{document}
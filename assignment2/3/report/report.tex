\documentclass[a4paper]{article}

\usepackage[utf8]{inputenc}
\usepackage{microtype}
\usepackage{enumitem}
\usepackage{comment}
\usepackage{float, graphicx}
\usepackage{mathtools, amssymb}
\usepackage{caption}
\usepackage{subcaption}

\setlength{\parindent}{0em}


\title{3}
\date{}

\begin{document}
\maketitle




Let I be the original image with a given pdf distribution $P_I(i)$.
The addition of Gaussian noise can be treated as the addition of an image J to I where the pdf of J is a Gaussian.
Let the resultant image be K with pdf $P_K(k)$. We have:

\begin{equation}
    P_K(k) = \int_{-\infty}^{\infty} P_{I,J}(i,k-i)di
\end{equation}

Which is nothing but the convolution of I and J

Since it is mentioned that the additive noise is applied independently, we can assume that the pdf of I and J are independent. Thus,

\begin{equation}
    P_K(k) = \int_{-\infty}^{\infty} P_{I}(i)P_{J}(k-i)di   
\end{equation}

The form for the pdf of J is,
\begin{equation}
P_J(j) = \frac{e^{\frac{-j^2}{2\sigma^2}}}{\sqrt{2\pi\sigma^2}}  
\end{equation}

Hence, the expression for the pdf of the Final image is,

\begin{equation}
    P_K(k) = \int_{-\infty}^{\infty} P_{I}(i) \frac{e^{\frac{-(k-i)^2}{2\sigma^2}}}{\sqrt{2\pi\sigma^2}} di   
\end{equation}




\end{document}


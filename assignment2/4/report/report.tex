\documentclass[a4paper]{article}

\usepackage[utf8]{inputenc}
\usepackage{microtype}
\usepackage{enumitem}
\usepackage[DIV=9]{typearea}
\usepackage{comment}
\usepackage{physics}
\usepackage{float, graphicx}
\usepackage{mathtools, amssymb}
\usepackage{caption}
\usepackage{amsmath}
\usepackage{subcaption}
\usepackage{enumitem}
\setlength{\parindent}{0em}


\title{4}
\date{}

\begin{document}
\maketitle
\begin{enumerate}[label=(\alph*)]
    \item 
Given a 2D operator defined by a matrix $G \in R^{n\times n}$, using SVD decomposition we can write it as:
\begin{equation}
    G = \sum_{i = 1}^{n} \sigma_{i}u_{i}v_{i}^T
\end{equation}
Clearly, G is separable iff $\forall i > 1, \sigma_i = 0$.
Hence the number of non-zero singular values should be equal to 1. Since that is equal to the rank of the matrix, the rank of the matrix should be 1.

The given matrix$
\begin{bmatrix}
    0 & 1 & 0 \\
    1 & -4 & 1\\
    0 & 1 & 0 
\end{bmatrix}
$, clearly has rank 2. Hence, it can not be realised as a separable filter.

\item Since the Matrix is not separable, it cannot be written as an outer product of 2,1-D vectors. Hence you cannot separate this kernel and apply 1-D convolutions consecutively to get the same result. However, the given Laplacian operator is written as,
\begin{align*}
    & f(x+1,y) + f(x-1,y) + f(x,y+1) + f(x,y-1) - 4f(x,y)\\
    & = \underbrace{(f(x+1,y) + f(x-1,y) -2f(x,y))}_{\frac{\partial^2 f}{\partial x^2}} + \underbrace{(f(x,y+1) + f(x,y-1) - 2f(x,y))}_{\frac{\partial^2 f}{\partial y^2}}
\end{align*}
We can make 2\textsuperscript{nd} order derivatives w.r.t $x$ and $y$ using 1D convolution with the kernels
$ \begin{bmatrix}1&-2&1 \end{bmatrix} $ \&
$ \begin{bmatrix}1&-2&1 \end{bmatrix}^T $ and sum their results to get the same result as applying the Laplacian mask.
Thus, the Laplacian mask can be implemented entirely using 1D convolutions.
\end{enumerate}
\end{document}
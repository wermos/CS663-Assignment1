\documentclass[a4paper]{article}

\usepackage[utf8]{inputenc}
\usepackage{microtype}
\usepackage{enumitem}
\usepackage{comment}
\usepackage{float, graphicx}
\usepackage{mathtools, amssymb}
\usepackage{caption}
\usepackage{subcaption}

\setlength{\parindent}{0em}


\title{5}
\date{}

\begin{document}
\maketitle

Let I denote the image and M denote the mean filter. Applying the mean is equivalent to a convolution between image I and filter M. 

\begin{equation}
    f_1 = I * M 
\end{equation}

Now, we apply the filter again, leading to another convolution

\begin{align}
     f_2 & = (I*M)*M \\
         & = I*(M*M) \\
\end{align}
Hence $f_k$ is given by

\begin{equation}
    f_k = I*(M*M*...k-times)
\end{equation}

If k is large enough, we can apply the Central Limit Theorem which states that if you have a bunch of distributions $f_i$ and you convolve them all together into a distribution $F:= f_1*f_@*f_3*..f_k$, then the larger k is, the more F will resemble a Gaussian distribution. Hence the above equation is equivalent to convolving the the initial distribution with a Gaussian.

\begin{equation}
    f_k = I*\mathcal{G}
\end{equation}

Also, it can be noted that if the above process is repeated multiple times, it would be equivalent to taking a convolution of the image with a Gaussian multiple times. Which is equivalent to taking a convolution with a Gaussian of a larger $\sigma$. Hence the resultant image will ultimately lead to having the same average intensity at each pixel.
    






\end{document}


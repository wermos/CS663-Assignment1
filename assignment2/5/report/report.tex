\documentclass[a4paper]{article}

\usepackage[utf8]{inputenc}
\usepackage{microtype}
\usepackage{enumitem}
\usepackage{comment}
\usepackage{float, graphicx}
\usepackage{mathtools, amssymb}
\usepackage{caption}
\usepackage{subcaption}

\setlength{\parindent}{0em}


\title{5}
\date{}

\begin{document}
\maketitle

Let $I$ denote the image and $M$ denote the mean filter. Applying the mean is equivalent to a convolution between image $I$ and filter $M$.

\begin{equation}
    f_1 = M \ast I 
\end{equation}

Now, we apply the filter again, leading to another convolution

\begin{align}
     f_2 & = M \ast (M \ast I) &\\
         & = (M \ast M) \ast I &\text{(associativity of convolution)}\\
\end{align}
Hence, $f_k$ is given by

\begin{align}
    f_k &= M \ast f_{k-1}\\
        &= M \ast ( \underbrace{M \ast M \cdots M \ast M}_{k-1 \text{ $M$'s}} \ast I )\\
    f_k    &= (\underbrace{M \ast M \cdots M \ast M}_{k \text{ $M$'s}}) \ast I 
\end{align}
\begin{equation}
    \text{Now, let } N=\underbrace{M \ast M \cdots M \ast M}_{k \text{ $M$'s}} \Rightarrow f_k = N\ast I
\end{equation}
Thus, $N$ is a filter that gives the same result when applied to $I$ as a mean filter when applied to $I$, $k$ times.


Note that if $k$ is large enough, we can apply the Central Limit Theorem which states that if you have a bunch of distributions $f_i$ and you convolve them all together into a distribution $F:= f_1\ast f_2\ast f_3\ast \cdots f_k$, then the larger $k$ is, the more $F$ will resemble a Gaussian distribution. Hence in the limit, the above equation is equivalent to convolving the the initial distribution with a Gaussian.

\begin{equation}
    \lim_{k\rightarrow\infty}f_{k} = I*\mathcal{G}
\end{equation}
% Also, it can be noted that if the above process is repeated multiple times, it would be equivalent to taking a convolution of the image with a Gaussian multiple times. Which is equivalent to taking a convolution with a Gaussian of a larger $\sigma$. Hence the resultant image will ultimately lead to having the same average intensity at each pixel.
\end{document}
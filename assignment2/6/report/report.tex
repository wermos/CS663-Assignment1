\documentclass[a4paper]{article}

\usepackage{microtype}
\usepackage{mathtools, amssymb}
\usepackage{hyperref}

\hypersetup{
    colorlinks=true,
    urlcolor=blue
}

\newcommand{\R}{\mathbb{R}}

\setlength{\parindent}{0em}

\begin{document}
\begin{enumerate}
\item For this part, we have to derive an expression for $I(x)$ (where $I$ is a 1D ramp image) after filtering it by a zero-mean Gaussian filter with standard deviation $\sigma$. In other words, we convolve $g$ with $I$, where $g$ is the PDF associated with $N(0, \sigma^2)$ (the zero-mean Gaussian distribution). So, we get
\begin{align*}
J(x) &= (g * I)(x) \\
&= \int\limits_{\mathclap{-\infty}}^{\mathclap{\infty}}\! g(y) I (x - y)\, \mathrm{d}y \\
&= \int\limits_{\mathclap{-\infty}}^{\mathclap{\infty}}\! g(y)\left ( c(x - y) + d\right )\, \mathrm{d}y \\
&= \underbrace{-\int\limits_{\mathclap{-\infty}}^{\mathclap{\infty}}\! c y g(y)\, \mathrm{d}y}_{\mathcal{I}_1} + \int\limits_{\mathclap{-\infty}}^{\mathclap{\infty}}\! (cx + d)g(y)\, \mathrm{d}y
\end{align*}

In the above expression for $J(x)$, we observe that $\mathcal{I}_1$ is 0 because the integrand is an odd function. Therefore, upon simplifying, we get
\begin{align}
J(x) &= \int\limits_{\mathclap{-\infty}}^{\mathclap{\infty}}\! (cx + d)g(y)\, \mathrm{d}y \nonumber \\
&=  (cx + d)\int\limits_{\mathclap{-\infty}}^{\mathclap{\infty}}\!g(y)\, \mathrm{d}y \label{eq1} \\
&= cx + d \label{eq2}
\end{align}
where \eqref{eq1} follows because $c$ and $d$ are constants in $\R$, and $x$ is constant w.r.t. $y$, and \eqref{eq2} follows because the integral of a PDF over the entire space is equal to 1.

\item Note that the general definition of a bilateral filter (as can be found \href{https://en.wikipedia.org/wiki/Bilateral_filter#Definition}{here}, for example) does not require the intensity smoothing filter and the distance smoothing filter to be Gaussian filters. For this question, we assume that both filters are Gaussian filters, in accordance to what was taught in class and what was shown in the slides.

Let the bilateral Gaussian filter with parameters $\sigma_s$ and $\sigma_r$, be denoted by $b(x)$. Similar to the previous question, in order to find an expression for $J(x)$, we will convolve $b$ with $I$ to get the new image. We get
\begin{align*}
J(x) &= (b * I)(x) \\
&= \int\limits_{\mathclap{-\infty}}^{\mathclap{\infty}}\! b(y) I (x - y)\, \mathrm{d}y \\
&= \int\limits_{\mathclap{-\infty}}^{\mathclap{\infty}}\! b(y)\left ( c(x - y) + d\right )\, \mathrm{d}y
\end{align*}

Upon expanding $b(y)$, we get
\begin{align*}
J(x) &= \int\limits_{\mathclap{-\infty}}^{\mathclap{\infty}}\! (cx - cy + d) \left ( \dfrac{\exp\left (\frac{-y^2}{2\sigma_s^2}\right )}{\sqrt{2\pi}\sigma_s} \right ) \left ( \dfrac{\exp\left (\frac{-(I(x - y) - I(x))^2}{2\sigma_r^2}\right )}{\sqrt{2\pi}\sigma_r} \right )\dfrac{1}{W} \, \mathrm{d}y
\end{align*}
where
\begin{equation*}
W = \int\limits_{\mathclap{-\infty}}^{\mathclap{\infty}}\! b(y)\, \mathrm{d}y
\end{equation*}



\end{enumerate}
\end{document}
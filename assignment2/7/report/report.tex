\documentclass[a4paper]{article}

\usepackage[utf8]{inputenc}
\usepackage{microtype}
\usepackage{mathtools, amssymb, amsthm}
\usepackage{hyperref}

\hypersetup{
    colorlinks=true,
    linkcolor=blue,
    filecolor=magenta,
    urlcolor=blue
}

\setlength{\parindent}{0em}

\newcommand{\R}{\mathbb{R}}

%————————————————Definitions for the \pd command—————————————————————

%pd is short for ``partial derivative". The non-starred variant (which is meant to be used more often) takes two inputs: One for the function being differentiated, and the other is the variable we are differentiating with respect to.
\makeatletter
\newcommand{\pd}{\@ifstar{\@pdd}{\@pdn}}
\newcommand{\@pdd}[1]{\dfrac{\partial}{\partial #1}}
\newcommand{\@pdn}[2]{\dfrac{\partial #1}{\partial #2}}
\makeatother
%————————————————————————————————————————————————————————————————————

%——————————————————Definitions for the \spd command——————————————————

%spd is short for ``second partial derivative". The non-starred variant (which is meant to be used more often) takes two inputs: One for the function being differentiated, and the other is the variable we are differentiating with respect to.
\makeatletter
\newcommand{\spd}{\@ifstar{\@spdd}{\@spdn}}
\newcommand{\@spdd}[1]{\dfrac{\partial^2}{\partial #1^2}}
\newcommand{\@spdn}[2]{\dfrac{\partial^2 #1}{\partial #2^2}}
\makeatother
%————————————————————————————————————————————————————————————————————

%——————————————————Definitions for the \mspd command——————————————————

%spd is short for ``mixed second partial derivative". The non-starred variant (which is meant to be used more often) takes two inputs: One for the function being differentiated, and the other is the variable we are differentiating with respect to.
\makeatletter
\newcommand{\mspd}{\@ifstar{\@mspdd}{\@mspdn}}
\newcommand{\@mspdd}[2]{\dfrac{\partial^2}{\partial #1 \partial #2}}
\newcommand{\@mspdn}[3]{\dfrac{\partial^2 #1}{\partial #2 \partial #3}}
\makeatother
%————————————————————————————————————————————————————————————————————

\title{7}
\date{}

\begin{document}
\maketitle

Let $\Omega$ be an open set in $\R^n$ for some $n$, and let $f\colon \Omega \to \R$ be a function such that $f\in C^2(\Omega)$ (i.e., $f$ is twice-continuously differentiable).

\medskip
Let
\begin{align*}
u &= x\cos\theta - y\sin\theta \\
v &= x\sin\theta + y\cos\theta
\end{align*}
be a rotation of the coordinate system by $\theta$.

\medskip
To show that the Laplacian is rotationally invariant, it suffices to show that $f_{xx} + f_{yy}$ is equal to $f_{uu} + f_{vv}$ for any arbitrary $f$.

\smallskip
We know that
\begin{equation*}
\pd{f}{x} = \pd{f}{u}\pd{u}{x} + \pd{f}{v}\pd{v}{x}
\end{equation*}

Substituting the expressions for $\pd{u}{x}$ and $\pd{v}{x}$, we get
\begin{equation*}
\pd{f}{x} = \pd{f}{u}\cos\theta + \pd{f}{v}\sin\theta
\end{equation*}

We need to differentiate the above expression w.r.t. $x$ once more to get $\spd{f}{x}$:
\begin{align*}
\spd{f}{x} &= \pd*{x} \left (\pd{f}{x} \right ) \\
&= \pd*{x} \left ( \pd{f}{u} \cos\theta + \pd{f}{v} \sin\theta \right ) \\
&= (\cos\theta) \underbrace{\pd*{x}\left ( \pd{f}{u} \right )}_{T_1} + (\sin\theta) \underbrace{\pd*{x}\left ( \pd{f}{v} \right )}_{T_2}
\end{align*}

Now, we will calculate $T_1$ and $T_2$ separately:
\begin{align*}
T_1 = \pd*{x}\left ( \pd{f}{u} \right )
&= \pd*{u}\left ( \pd{f}{u} \right ) \pd{u}{x} + \pd*{v} \left ( \pd{f}{u} \right ) \pd{v}{x} \\
&= \spd{f}{u} \pd{u}{x} + \mspd{f}{v}{u} \pd{v}{x} \\
&= \spd{f}{u} \cos\theta + \mspd{f}{v}{u} \sin\theta
\end{align*}

\begin{align*}
T_2 = \pd*{x}\left ( \pd{f}{v} \right )
&= \pd*{u}\left ( \pd{f}{v} \right ) \pd{u}{x} + \pd*{v} \left ( \pd{f}{v} \right ) \pd{v}{x} \\
&= \mspd{f}{u}{v} \pd{u}{x} + \spd{f}{v} \pd{v}{x} \\
&= \mspd{f}{u}{v} \cos\theta + \spd{f}{v} \sin\theta
\end{align*}

Putting it all together, we get
\begin{equation*}
\spd{f}{x} = (\cos\theta) \left ( \spd{f}{u} \cos\theta + \mspd{f}{v}{u} \sin\theta \right ) + (\sin\theta) \left ( \mspd{f}{u}{v} \cos\theta + \spd{f}{v} \sin\theta \right )
\end{equation*}

Since our hypothesis is that $f$ is a $C^2$ function, we can use \href{https://en.wikipedia.org/wiki/Symmetry_of_second_derivatives#Schwarz's_theorem}{Schwarz's theorem} to simplify the above expression further:
\begin{equation}
\spd{f}{x} = \spd{f}{u} \cos^2\theta + 2\mspd{f}{u}{v} \sin\theta \cos\theta + \spd{f}{v} \sin^2\theta
\label{eq1}
\end{equation}

Now, we will perform the same steps to compute $\spd{f}{y}$, starting with the expression for $\pd{f}{y}$:
\begin{equation*}
\pd{f}{y} = \pd{f}{u}\pd{u}{y} + \pd{f}{v}\pd{v}{y} = -\pd{f}{u}\sin\theta + \pd{f}{v}\cos\theta
\end{equation*}

From this, we get
\begin{align*}
\spd{f}{y} &= \pd*{y} \left (\pd{f}{y} \right ) \\
&= \pd*{y} \left ( -\pd{f}{u}\sin\theta + \pd{f}{v}\cos\theta \right) \\
&= (-\sin\theta) \underbrace{\pd*{y}\left ( \pd{f}{u} \right )}_{T_3} + (\cos\theta) \underbrace{\pd*{y}\left ( \pd{f}{v} \right )}_{T_4}
\end{align*}

Now, we will calculate $T_3$ and $T_4$ separately:
\begin{align*}
T_3 = \pd*{y}\left ( \pd{f}{u} \right )
&= \pd*{u}\left ( \pd{f}{u} \right ) \pd{u}{y} + \pd*{v} \left ( \pd{f}{u} \right ) \pd{v}{y} \\
&= \spd{f}{u} \pd{u}{y} + \mspd{f}{v}{u} \pd{v}{y} \\
&= -\spd{f}{u} \sin\theta + \mspd{f}{v}{u} \cos\theta
\end{align*}

\begin{align*}
T_4 = \pd*{y}\left ( \pd{f}{v} \right )
&= \pd*{u}\left ( \pd{f}{v} \right ) \pd{u}{y} + \pd*{v} \left ( \pd{f}{v} \right ) \pd{v}{y} \\
&= \mspd{f}{u}{v} \pd{u}{y} + \spd{f}{v} \pd{v}{y} \\
&= -\mspd{f}{u}{v} \sin\theta + \spd{f}{v} \cos\theta
\end{align*}

Hence,
\begin{equation*}
\spd{f}{y} = -\sin\theta \left ( -\spd{f}{u} \sin\theta + \mspd{f}{v}{u} \cos\theta \right ) + \cos\theta \left ( -\mspd{f}{u}{v} \sin\theta + \spd{f}{v} \cos\theta \right )
\end{equation*}

By using Schwarz's Theorem again, we can simplify the above expression to
\begin{equation}
\spd{f}{y} = \spd{f}{u} \sin^2\theta -2\mspd{f}{u}{v} \sin\theta\cos\theta + \spd{f}{v} \cos^2\theta
\label{eq2}
\end{equation}

By adding \eqref{eq1} and \eqref{eq2}, we get
\begin{align*}
\spd{f}{x} + \spd{f}{y} &= \left ( \spd{f}{u} \cos^2\theta + 2\mspd{f}{u}{v} \sin\theta \cos\theta + \spd{f}{v} \sin^2\theta \right ) \\
&\phantom{{}={}} + \left ( \spd{f}{u} \sin^2\theta -2\mspd{f}{u}{v} \sin\theta\cos\theta + \spd{f}{v} \cos^2\theta \right ) \\
&= \spd{f}{u} \left ( \cos^2\theta + \sin^2\theta \right ) + \spd{f}{v} \left ( \sin^2\theta + \cos^2\theta \right ) \\
&= \spd{f}{u} + \spd{f}{v} \qed
\end{align*}




\end{document}
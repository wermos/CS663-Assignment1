\documentclass[a4paper]{article}

\usepackage{microtype}
\usepackage{enumitem}
\usepackage{mathtools, amssymb, amsthm}
\usepackage{bbm}
\usepackage{nicematrix}

\setlength{\parindent}{0em}

\title{6}
\date{}

\newcommand{\F}{\mathcal{F}}
\newcommand{\R}{\mathbb{R}}

\newcommand{\infint}{\int\limits_{\mathclap{-\infty}}^{\mathclap{\infty}}}\renewcommand{\d}{\mathrm{d}}

\begin{document}
\maketitle

Recall that, given a continuous function $f\colon \R \to \R$, the Fourier Transform of $f$ is given by
\begin{equation*}
\hat{f}(\xi) \coloneq \F\{f(t)\} = \infint \! f(t) e^{-i2\pi \xi t} \, \d t,
\end{equation*}
and the inverse Fourier Transform is given by
\begin{equation*}
f(x) = \infint \! \hat{f}(\xi) e^{i2\pi \xi t} \, \d \xi,
\end{equation*}

Throughout this answer, we will denote the Fourier Transform operator by $\F$.

\bigskip

As the hint suggests, we will first investigate what we get when we simplify $\F\{\F\{f(t)\}\}$. Since $\hat{f}(\xi) = \F\{f(t)\}$, we know that $\F\{\F\{f(t)\}\} = \F\{\hat{f}(\xi)\}$. Expanding the expression, we get 
\begin{equation*}
\F\{\hat{f}(\xi)\} = \infint \hat{f}(\xi) e^{-i2\pi \xi t} \d \xi
\end{equation*}

By performing a change of variable with $u = -\xi$ (so that $\d u = -\d \xi$), we get
\begin{align*}
\infint \hat{f}(\xi) e^{-i2\pi \xi t} \d \xi &= -\int\limits_{\mathclap{\infty}}^{\mathclap{-\infty}} \hat{f}(-u) e^{i2\pi u t} \d u \\
&= \infint \hat{f}(-u) e^{i2\pi u t} \d u
\end{align*}

The last expression is, by definition, the inverse Fourier Transform of $f(-u)$. Hence, we have shown (sans some variable renaming) that
\begin{equation*}
\F\{\F\{f(t)\}\} = f(-t)
\end{equation*}

With this lemma in mind, we can easily prove the given proposition. Given $\F\{\F\{\F\{\F\{f(t)\}\}\}\}$, we can simplify it to $\F\{\F\{f(-t)\}\}$ by using the previous lemma. By applying the same lemma again, we get$\F\{\F\{f(-t)\}\} = f\big(-(-t)\big) = f(t)$, which proves the proposition. \qed


\end{document}
\documentclass[a4paper]{article}

\usepackage[utf8]{inputenc}
\usepackage{microtype}
\usepackage{enumitem}
\usepackage{comment}
\usepackage{float, graphicx}
\usepackage{amsmath}
\usepackage{mathtools, amssymb, mathrsfs}
\usepackage{caption}
\usepackage{subcaption}

\setlength{\parindent}{0em}


\title{3}
\date{}

\begin{document}
\maketitle


\section{3a}
The goal is to maximize $f^{T}Cf$ constrained to the condition that $f \perp e$ and $\lvert f \rvert = 1$

The Lagrangian for this optimization problem can be written as:
\[\mathscr{L} = f^{T}Cf + \alpha_{2}(1 - \lvert f \rvert) + \beta(0 - f^{T}e)\]
\[ =  f^{T}Cf + \alpha_{2}(1 - f^{T}f) + \beta(0 - f^{T}e) \]

Differentiating the above equation w.r.t $f$, we get:
\begin{equation}
 2Cf - 2\alpha_{2}(f)  -\beta(e) = 0 \label{eq:1}
\end{equation}

Left multiplying the equation by $e^{T}$:
\[2e^{T}Cf - 2\alpha_{2}e^{T}f - \beta e^{T}e = 0\]

Since e is an eigenvector of C (say, with eigenvalue $\alpha_{1}$), we have:
\[ Ce = \alpha_{1}e\]
or, transposing both sides:
\[e^{T}C = \alpha_{1}e^{T}\]
Hence we obtain:
\[2\alpha_{1}e^{T}f - 2\alpha_{2}e^{T}f - \beta e^{T}e = 0\]
Clearly,$f \perp e \implies  e^{T}f = 0$ and $e^{T}e = 1$, hence:
\[\beta = 0\]
Equation $\eqref{eq:1}$ reduces to:
\[Cf = \alpha_{2}f\]
Hence, we conclude that $f$ is an eigenvector of $C$. In order to maximize $f^{T}Cf$ we need to maximize $\alpha_2$. Because $f$ must be orthogonal to $e$, it cannot be the eigenvector corresponding to the largest eigenvalue($\alpha_{2} \neq \alpha_{1}$). As such, $f$ must be an eigenvector corresponding to the second-largest eigenvalue.
\section{3b}
Let us now consider the expression $g^{T}Cg$ where $g$ is a vector perpendicular to both $f$ and $e$. The Lagrangian for this optimization problem can be written as:
\[\mathscr{L} = g^{T}Cg + \alpha_{3}(1 - \lvert g \rvert) + \beta(0 - g^{T}e) + \gamma(0 - g^{T}f )\]
\[ = g^{T}Cg + \alpha_{3}(1 - g^{T}g) + \beta(0 - g^{T}e) + \gamma(0 - g^{T}f )) \]
Differentiating the above equation w.r.t $g$, we get:
\begin{equation}
 2Cg - 2\alpha_{3}(g)  -\beta(e) - \gamma(f) = 0 \label{eq:2}
\end{equation}

Left multiplying the equation by $e^{T}$:
\[2e^{T}Cg - 2\alpha_{3}e^{T}g - \beta e^{T}e - \gamma e^{T}f = 0\]
As seen before, $e^{T}Cg = \alpha_{1}e^{T}g = 0$. Also, $e^{T}g = 0$ and $e^{T}f = 0$. Hence:
\[\beta = 0\]

Similarly, we can left multiply  $\eqref{eq:2}$ by $f^{T}$ to obtain:
\[2f^{T}Cg - 2\alpha_{3}f^{T}g - \beta f^{T}e - \gamma f^{T}f = 0\]
Again, $f^{T}Cg = \alpha_{2}f^{T}g = 0$. Also, $f^{T}g = 0$ and $f^{T}e = 0$. Hence:
\[\gamma = 0\]
Thus, $\eqref{eq:2}$ reduces to:
\[Cg = \alpha_{3}g\]
Hence, we conclude that $g$ is an eigenvector of $C$. In order to maximize $g^{T}Cg$ we need to maximize $\alpha_3$. Because $g$ must be orthogonal to $e$ and $f$, it cannot be the eigenvector corresponding to the largest eigenvalue($\alpha_{3} \neq \alpha_{1}$) or the second-largest eigenvalue($\alpha_{3} \neq \alpha_{2}$). As such, $f$ must be an eigenvector corresponding to the third-largest eigenvalue.

He can proceed inductively to prove the result for the $j^{th}$ orthogonal vector.

\end{document}


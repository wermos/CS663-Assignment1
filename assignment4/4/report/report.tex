\documentclass[a4paper]{article}

\usepackage[DIV=12]{typearea}
\usepackage{microtype}
\usepackage[shortlabels]{enumitem}
\usepackage{mathtools, amssymb}
\usepackage{bm}
\setlength{\parindent}{0em}
\title{5}
\date{}

\begin{document}
\maketitle
\begin{enumerate}[(a)]
\item 
Covariance matrix in PCA is given as follows
\begin{equation}
	\bm{C} = \frac{1}{N-1}\bm{X}\bm{X}^T \text{ where $\bm{X} = \begin{bmatrix} \overline{\bm{x}}_1 & \cdots & \overline{\bm{x}}_N \end{bmatrix}$ and $\overline{\bm{x}}\in \mathbb{R}^d$}
\end{equation}
Now, this covariance matrix is symmetric as
\begin{equation}
\begin{aligned}
	\bm{C}^T &= \left(\frac{1}{N-1}\bm{X}\bm{X}^T\right)^T\\
	& = \frac{1}{N-1}(\bm{X}^T)^T \bm{X}^T \\	
	& = \frac{1}{N-1}\bm{X}\bm{X}^T \\	
	& = \bm{C} \\	
\end{aligned}
\end{equation}
Now, to prove that $\bm{C}$ is positive semi-definite we assume the opposite and show its infeasibility. That is, assume it is not positive semi-definite then there exists a vector $\bm{y}\in\mathbb{R}^d$ such that $\bm{y}^T\bm{C}\bm{y}<0$.
\begin{equation}
 	\begin{aligned}
 		\text{Now, }\exists \bm{y}\in\mathbb{R}^d | \bm{y}^T\bm{C}\bm{y}<0&\Leftrightarrow\bm{y}^T\frac{1}{N-1}\bm{X}\bm{X}^T\bm{y}<0\\
 		&\Leftrightarrow \frac{1}{N-1}\bm{y}^T\bm{X}\bm{X}^T\bm{y}<0\\
 		&\Leftrightarrow \frac{1}{N-1}(\bm{X}^T\bm{y})^T\bm{X}^T\bm{y}<0\\
 		&\Leftrightarrow \frac{1}{N-1}\|\bm{X}^T\bm{y}\|<0\\
 		&\Leftrightarrow \|\bm{X}^T\bm{y}\|<0\\
 		&\Leftrightarrow \bot
 	\end{aligned}
\end{equation}
a contradiction.
\item 
Let $\bm{v}_1, \bm{v}_2$ be two eigenvectors of $\bm{A}$ with distinct eigenvalues $\lambda_1, \lambda_2$ respectively.
Now, consider
\begin{equation}
	\begin{aligned}
		\lambda_1 \bm{v}_1^T \bm{v}_2
		&= \bm{A} \bm{v}_1^T \bm{v}_2\\
		&= \bm{A} \bm{v}_2^T \bm{v}_1 & \hfill{(\text{$\bm{v}_1^T \bm{v}_2$ is a scalar, so it is equal to its transpose})}\\
		&= \lambda_2 \bm{v}_2^T \bm{v}_1\\
		&= \lambda_2 \bm{v}_1^T \bm{v}_2 & \hfill{(\text{taking transpose again})}\\
	\end{aligned}
\end{equation}
This implies
\begin{equation}
	\begin{aligned}
		&\lambda_1 \bm{v}_1^T \bm{v}_2 - \lambda_2 \bm{v}_1^T \bm{v}_2 = 0\\
		&\Leftrightarrow(\lambda_1 - \lambda_2) \bm{v}_1^T \bm{v}_2 = 0\\
		&\Leftrightarrow\bm{v}_1^T \bm{v}_2 = 0 & \hfill{(\text{since eigenvalues are distinct})}
	\end{aligned}
\end{equation}
Hence, the eigenvectors of a symmetric matrix are orthonormal.
\item 
\item 
\end{enumerate}
\end{document}
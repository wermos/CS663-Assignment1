\documentclass[a4paper]{article}

\usepackage[utf8]{inputenc}
\usepackage{microtype}
\usepackage{enumitem}
\usepackage{comment}
\usepackage{float, graphicx}
\usepackage{amsmath}
\usepackage{mathtools, amssymb, mathrsfs}
\usepackage{caption}
\usepackage{subcaption}

\setlength{\parindent}{0em}


\title{2}
\date{}

\begin{document}
\maketitle
\section{Fourier-based Translation Approach}
In the paper, a Fourier domain-based approach is presented to find the translation between two images. Let us assume that we have two images $f_1(x,y)$ and $f_2(x,y)$
with:
\[f_2(x,y) = f_1(x-x_0,y-y_0)\]

Because of the Fourier shift theorem, the Fourier transforms of these images will differ only by a phase factor:
\[\mathscr{F}_2(x',y') = \mathscr{F}_1(x',y') e^{-j2\pi(x'x_0) + y'y_0))}\]

To separate out the phase factor, we then calculate the cross-power spectrum as:
\[\frac{\mathscr{F_1}(x',y') \mathscr{F_2}(x',y')^*}{\|\mathscr{F_1)}(x',y'\mathscr{F_2}(x',y') \|} = e^{j2 \pi (x'x_0 + y'y_0 )}\]

The IFT of the above function is simply $\delta(x-x_0, y-y_0)$. Plotting this function will provide us with a clear spike at the displacement $(x_0, y_0)$.
Thus we can find the displacement between the two images that have a translation.

\subsection{Time Complexity of this Approach}
This procedure involves the calculation of two FFTs for image $f_1(x,y)$ and $f_2(x,y)$. And one IFFT of the cross-power spectrum.
The time complexity of a single FFT/IFFT of an $N \times N$ image is:
\[T = O(N^2 \operatorname{log}(N^2)) = O(N^2 \operatorname{log}(N)) \]
Thus, the time complexity of this procedure is:
\[T = O(3N^2 \operatorname{log}(N)) = O(N^2 \operatorname{log}(N)) \]

\subsection{Comparison of Complexity with Pixel-based Methods}
We will compare this to the MSSD method that we studied in the early weeks of this course. In the procedure, we iterate the translation parameters $t_x$ and $t_y$ over a search range(say $M_1$ for $t_x$ and $M_2$ for $t_y$). For every such pair, we have a Transformation matrix and we transform image 2 using it. Finally, we calculate the MSSD of these two images and pick the transformation matrix that corresponds to the least MSSD value. Here, we are using the brute-force method.
The time complexity of Transforming an $N\times N$ image is $O(N^2)$
The time complexity of MSSD calculation is $O(N^2)$. Thus total complexity of the procedure is $O(M_1*M_2*N^2*N^2) = O(M_1M_2N^4)$.
We can clearly see that the Fourier-based method is way faster and more precise.

\section{Correcting for Rotation}
In order to find the rotation between two images, let us consider images $f_1(x,y$  and $f_2(x,y)$ with:
\[f_2(x,y) = f_1(x\operatorname{cos}(\theta_0) + y\operatorname{sin}(\theta_0) - x_0,-x\operatorname{sin}(\theta_0) + y\operatorname{cos}(\theta_0) - y_0)\]

The Fourier transform of these images is given by:
\[\mathscr{F}_2(x',y') = \mathscr{F}_1(x'\operatorname{cos}(\theta_0) + y'\operatorname{sin}(\theta_0),-x'\operatorname{sin}(\theta_0) + y'\operatorname{cos}(\theta_0)) e^{-j2\pi(x'x_0) + y'y_0))}\]

Now, if we only look at the magnitudes of $\mathscr{F}_1$ and $\mathscr{F}_2$, let them be $M_1$ and $M_2$:

\[M_2(x',y') = M_1(x'\operatorname{cos}(\theta_0) + y'\operatorname{sin}(\theta_0),-x'\operatorname{sin}(\theta_0) + y'\operatorname{cos}(\theta_0)) \]

Hence, one is the rotated replica of the other. If we represent this magnitude in polar coordinates, then we can see this rotation as a translation in $\theta$, ie.:

\[M_1(\rho, \theta) = M_2(\rho, \theta - \theta_0)\]
Hence, we can apply the phase correlation procedure elaborated in section 1 to find out this translation in $\theta$ and make a correction for rotation.

If the two images have different scaling, their Fourier magnitude will have the following relation:
\[M_1(\rho, \theta) = M_2(\rho/a, \theta - \theta_0)\]
Hence, we can still employ phase correlation technique to find $\theta_0$





\end{document}
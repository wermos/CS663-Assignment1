\documentclass[a4paper, landscape]{article}

\usepackage[DIV=18]{typearea}
\usepackage[utf8]{inputenc}
\usepackage{microtype}
\usepackage{mathtools, amssymb}
\usepackage[colorlinks=true]{hyperref}
\renewcommand{\d}{\, \mathrm{d}}
\title{4}
\date{}

\setlength{\parindent}{0em}

\begin{document}
\maketitle
\section{Approach}{\label{sec:a}}
We are assuming the images are 2-D and carry out Fourier analysis in continuous domain.
For given equations,
\begin{equation}{\label{eq:given}}
	\begin{aligned}
		g_1(x,y) &= f_1(x,y) + h_2(x,y) \ast f_2(x,y)\\
		g_2(x,y) &= h_1(x,y) \ast f_1(x,y) + f_2(x,y)
	\end{aligned}
\end{equation}
Let $F_i, G_i, H_i$ be the Fourier transformations of $f_i, g_i, h_i$ respectively. After applying Fourier transformation on both sides to \ref{eq:given}, we get
\begin{subequations}
	\begin{align}
		{\label{eq:f1}}G_1(\mu,\nu) &= F_1(\mu,\nu) + H_2(\mu,\nu)F_2(\mu,\nu)\\
		{\label{eq:f2}}G_2(\mu,\nu) &= H_1(\mu,\nu)F_1(\mu,\nu) + F_2(\mu,\nu)
	\end{align}
\end{subequations}
Now, we can solve for $F_1$, $F_2$ by multiplying \ref{eq:f1} by $H_1$ and \ref{eq:f2} by $H_2$ then subtracting these equations from \ref{eq:given} and rearranging the terms
\begin{equation}
	% \begin{aligned}
	% 	G_1 &= F_1 + H_2 F_2\\
	% 	G_2 &= H_1F_1 + F_2\\
	% \end{aligned}
	% \text{\quad multiplying by $H_1$\quad}
	\begin{aligned}
		H_1G_1 &= H_1F_1 + H_1H_2 F_2\\
		H_2G_2 &= H_2H_1F_1 + H_2F_2
	\end{aligned}
	\quad\Rightarrow\quad
	\begin{aligned}
		H_1G_1 - (G_2) &= H_1F_1 + H_1H_2F_2 - (H_1F_1 + F_2)\\
		H_2G_2 - (G_1) &= H_2H_1F_1 + H_2F_2 - (F_1 + H_2F_2)
	\end{aligned}
	\quad\Rightarrow\quad
	\begin{aligned}
		H_1G_1 - G_2 &= H_1H_2F_2 - F_2\\
		H_2G_2 - G_1 &= H_2H_1F_1 - F_1
	\end{aligned}
	\quad\Rightarrow\quad
	\begin{aligned}
		H_1G_1 - G_2 &= F_2(H_1H_2 - 1)\\
		H_2G_2 - G_1 &= F_1(H_2H_1 - 1)
	\end{aligned}
\end{equation}
Now, we can get $f_1, f_2$ by taking inverse Fourier transform of $F_1, F_2$ respectively.
\begin{equation}{\label{eq:answer}}
	\begin{aligned}
		f_1(x,y) &= \mathcal{F}^{-1}\left(\frac{H_1G_1 - G_2}{H_1H_2 - 1}\right)(\mu,\nu)\\
		f_2(x,y) &= \mathcal{F}^{-1}\left(\frac{H_2G_2 - G_1}{H_1H_2 - 1}\right)(\mu,\nu)\\
	\end{aligned}
	\quad\text{where}\quad
	\left(
	\begin{aligned}
		F_1 (\mu,\nu) &= \frac{H_1 (\mu,\nu)G_1 (\mu,\nu) - G_2 (\mu,\nu)}{H_1 (\mu,\nu)H_2 (\mu,\nu) - 1}\\
		F_2 (\mu,\nu) &= \frac{H_2 (\mu,\nu)G_2 (\mu,\nu) - G_1 (\mu,\nu)}{H_1 (\mu,\nu)H_2 (\mu,\nu) - 1}\\
	\end{aligned}
	\quad\text{and}\quad
	\begin{aligned}
		% 1-D
		% f_1 &= \int_{-\infty}^{\infty}F_1(\mu)e^{j2\pi\mu t}\d\mu\\
		% f_2 &= \int_{-\infty}^{\infty}F_2(\mu)e^{j2\pi\mu t}\d\mu\\
		f_1(x,y) &= \int_{-\infty}^{\infty}\int_{-\infty}^{\infty}F_1(\mu, \nu)e^{j2\pi(\mu x+\nu y)}\d\mu\d\nu\\
		f_2(x,y) &= \int_{-\infty}^{\infty}\int_{-\infty}^{\infty}F_2(\mu, \nu)e^{j2\pi(\mu x+\nu y)}\d\mu\d\nu\\
	\end{aligned}
	\right)
\end{equation}
\section{Problem}
An issue can arise when the denominator in \ref{eq:answer} is zero, i.e., $H_1(\mu,\nu)H_2(\mu,\nu)=1$ for some $(\mu,\nu)$ . In this section we show that this is indeed possible.

Since the blur kernels preserves the brightness of the image, we can assume they are normalized, i.e.,
\begin{equation}
	\int_{-\infty}^{\infty}\int_{-\infty}^{\infty}h(x,y)\d x\d y = 1 \quad\text{and}\quad |H(\mu,\nu)|\leq1\ \forall (\mu,\nu)
\end{equation}
Under this assumption, we can see that $H_1(0,0)=H_2(0,0)=1$ as
\begin{equation}
	\begin{aligned}
		H(\mu, \nu)=\int_{-\infty}^{\infty}\int_{-\infty}^{\infty}h(x, y)e^{-j2\pi(\mu x+\nu y)}\d x\d y
		\quad\Rightarrow\quad
		H(0, 0)
		&=\int_{-\infty}^{\infty}\int_{-\infty}^{\infty}h(x, y)e^{-j2\pi(0 x+0 y)}\d x\d y\\
		&=\int_{-\infty}^{\infty}\int_{-\infty}^{\infty}h(x, y)e^{0}\d x\d y\\
		&=\int_{-\infty}^{\infty}\int_{-\infty}^{\infty}h(x, y)\d x\d y\\
		H(0, 0)&=1\\
	\end{aligned}
\end{equation}
So at $(\mu,\nu)=(0,0)$, both $F_1, F_2$ are undefined due to division by zero.

Hence, the formula given in \ref{sec:a} isn't foolproof.
\end{document}
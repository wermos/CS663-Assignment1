\title{Assignment 5: CS 663}
\author{}
\date{Due: 8th November before 11:55 pm}

\documentclass[11pt]{article}

\usepackage{amsmath}
\usepackage{amssymb}
\usepackage{hyperref}
\usepackage{ulem}
\usepackage[margin=0.4in]{geometry}
\begin{document}
\maketitle

\textbf{Remember the honor code while submitting this (and every other) assignment. You may discuss broad ideas with other students or ask me for any difficulties, but the code you implement and the answers you write must be your own. We will adopt a \textbf{zero-tolerance policy} against any violation.}
\\
\\
\textbf{Submission instructions:} Follow the instructions for the submission format and the naming convention of your files from the submission guidelines file in the homework folder. Please see \textsf{assignment5\_DFT.rar}. Upload the file on moodle \emph{before} 11:55 pm on 8th November \textbf{and this is also the cutoff deadline. This is different from other assignments where you had time till the next day 10 am.}.  Only one student per group needs to upload the assignment. No late assignments will be accepted after this time. Please preserve a copy of all your work until the end of the semester.  

\begin{enumerate}
\item In this part, we will apply the PCA technique for the task of image denoising. Consider the images \texttt{barbara256.png} and \texttt{stream.png} present in the corresponding data/ subfolder - this image has gray-levels in the range from 0 to 255. For the latter image, you should extract the top-left portion of size 256 by 256. Add zero mean Gaussian noise of $\sigma = 20$ to one of these images using the MATLAB code \texttt{im1 = im + randn(size(im))*20}. Note that this noise is image-independent. (If during the course of your implementation, your program takes too long, you can instead work with the file \texttt{barbara256-part.png} which has size 128 by 128 instead of 256 by 256. You can likewise extract the top-left 128 by 128 part of the \texttt{stream.png} image. You will not be penalized for working on these image parts.)
\begin{enumerate}
\item In the first part, you will divide the entire noisy image `im1' into overlapping patches of size 7 by 7, and create a matrix $\mathbf{P}$ of size $49 \times N$ where $N$ is the total number of image patches. Each column of $\mathbf{P}$ is a single patch reshaped to form a vector. Compute eigenvectors of the matrix $\mathbf{PP}^T$, and the eigen-coefficients of each noisy patch. 
Let us denote the $j^{\textrm{th}}$ eigen-coefficient of the $i^{\textrm{th}}$ (noisy) patch (i.e. $\mathbf{P}_i$) by $\alpha_{ij}$. Define $\bar{\alpha}^2_j = \textrm{max}(0,\frac{1}{N}[\sum_{i=1}^N \alpha^2_{ij}] - \sigma^2)$, which is basically an estimate of the average squared eigen-coefficients of the `original (clean) patches'. Now, your task is to manipulate the noisy coefficients $\{\alpha_{ij}\}$ using the following rule, which is along the lines of the Wiener filter update that we studied in class:
$\alpha^{\textrm{denoised}}_{ij} = \dfrac{\alpha_{ij}}{1 + \frac{\sigma^2}{\bar{\alpha}^2_j}}$.
Here, $\alpha^{\textrm{denoised}}_{ij}$ stands for the $j^{\textrm{th}}$ eigencoefficient of the $i^{\textrm{th}}$ denoised patch. Note that $\frac{\sigma^2}{\bar{\alpha}^2_j}$ is an estimate of the ISNR, which we absolutely need for any practical implementation of a Wiener filter update.  After updating the coefficients by the Wiener filter rule, you should reconstruct the denoised patches and re-assemble them to produce the final denoised image which we will call `im2'. Since you chose overlapping patches, there will be multiple values that appear at any pixel. You take care of this situation using simple averaging. Write a function \texttt{myPCADenoising1.m} to implement this. Display the final image `im2' in your report and state its RMSE computed as $\dfrac{\|\textrm{im2}_\textrm{denoised}-\textrm{im2}_\textrm{orig}\|_2}{\|\textrm{im2}_\textrm{orig}\|_2}$. 
\item In the second part, you will modify this technique. Given any patch $\mathbf{P}_i$ in the noisy image, you should collect $K = 200$ most similar patches (in a mean-squared error sense) from within a $31 \times 31$ neighborhood centered at the top left corner of $\mathbf{P}_i$. We will call this set of similar patches as $Q_i$ (this set will of course include $\mathbf{P}_i$). Build an eigen-space given $Q_i$ and denoise the eigen-coefficients corresponding to \textbf{only} $P_i$ using the Wiener update mentioned earlier. The only change will be that $\bar{\alpha}^2_j$ will now be defined using only the patches from $Q_i$ (as opposed to patches from all over the image). Reconstruct the denoised version of $P_i$. Repeat this for every patch from the noisy image (i.e. create a fresh eigen-space each time). At any pixel, there will be multiple values due to overlapping patches - simply average them. Write a function \texttt{myPCADenoising2.m} to implement this. Reconstruct the final denoised image, display it in your report and state the RMSE value. \textit{Do so for both barbara as well as stream.}

\item Now run your bilateral filter code from Homework 2 on the noisy version of the barbara image. Compare the denoised result with the result of the previous two steps for both images. What differences do you observe? What are the differences between this PCA based approach and the bilateral filter? 

\item Consider that a student clamps the values in the noisy image`im1' to the [0,255] range, and then denoises it using the aforementioned PCA-based filtering technique which assumes Gaussian noise. Is this approach correct? Why (not)? \textsf{[10 + 20 + 5 + 5 = 40 points]}
\end{enumerate}

\item Read Section 1 of the paper `An FFT-Based Technique for Translation, Rotation, and Scale-Invariant Image Registration' published in the IEEE Transactions on Image Processing in August 1996. A copy of this paper is available in the homework folder. 
\begin{enumerate}
\item Describe the procedure in the paper to determine translation between two given images. What is the time complexity of this procedure to predict translation if the images were of size $N \times N$? How does it compare with the time complexity of pixel-wise image comparison procedure for predicting the translation? 
\item Also, briefly explain the approach for correcting for rotation between two images, as proposed in this paper in Section II. Write down an equation or two to illustrate your point.
\end{enumerate} \textsf{[10+10=20 points]}

\item Consider a matrix $\boldsymbol{A}$ of size $m \times n, m \leq n$. Define $\boldsymbol{P} = \boldsymbol{A}^T \boldsymbol{A}$ and $\boldsymbol{Q} = \boldsymbol{A}\boldsymbol{A}^T$. (Note: all matrices, vectors and scalars involved in this question are real-valued).
\begin{enumerate}
\item Prove that for any vector $\boldsymbol{y}$ with appropriate number of elements, we have $\boldsymbol{y}^t \boldsymbol{Py} \geq 0$. Similarly show that $\boldsymbol{z}^t \boldsymbol{Qz} \geq 0$ for a vector $\boldsymbol{z}$ with appropriate number of elements. Why are the eigenvalues of $\boldsymbol{P}$ and $\boldsymbol{Q}$ non-negative?
\item If $\boldsymbol{u}$ is an eigenvector of $\boldsymbol{P}$ with eigenvalue $\lambda$, show that $\boldsymbol{Au}$ is an eigenvector of $\boldsymbol{Q}$ with eigenvalue $\lambda$. If $\boldsymbol{v}$ is an eigenvector of $\boldsymbol{Q}$ with eigenvalue $\mu$, show that $\boldsymbol{A}^T\boldsymbol{v}$ is an eigenvector of $\boldsymbol{P}$ with eigenvalue $\mu$. What will be the number of elements in $\boldsymbol{u}$ and $\boldsymbol{v}$?

\item If $\boldsymbol{v}_i$ is an eigenvector of $\boldsymbol{Q}$ and we define $\boldsymbol{u}_i \triangleq \dfrac{\boldsymbol{A}^T \boldsymbol{v}_i}{\|\boldsymbol{A}^T \boldsymbol{v}_i\|_2}$. Then prove that there will exist some real, non-negative $\gamma_i$ such that $\boldsymbol{Au}_i = \gamma_i \boldsymbol{v}_i$.

\item It can be shown that $\boldsymbol{u}^T_i \boldsymbol{u}_j = 0$ for $i \neq j$ and likewise $\boldsymbol{v}^T_i \boldsymbol{v}_j = 0$ for $i \neq j$ for correspondingly distinct eigenvalues. (You did this in HW4 where you showed that the eigenvectors of symmetric matrices are orthonormal.) Now, define $\boldsymbol{U} = [\boldsymbol{v}_1 | \boldsymbol{v}_2 | \boldsymbol{v}_3 | ...|\boldsymbol{v}_m]$ and $\boldsymbol{V} = [\boldsymbol{u}_1 | \boldsymbol{u}_2 | \boldsymbol{u}_3 | ... |\boldsymbol{u}_m]$. Now show that $\boldsymbol{A} = \boldsymbol{U} \boldsymbol{\Gamma} \boldsymbol{V}^T$ where $\boldsymbol{\Gamma}$ is a diagonal matrix containing the non-negative values $\gamma_1, \gamma_2, ..., \gamma_m$. With this, you have just established the existence of the singular value decomposition of any matrix $\boldsymbol{A}$. This is a key result in linear algebra and it is widely used in image processing, computer vision, computer graphics, statistics, machine learning, numerical analysis, natural language processing and data mining. \textsf[7.5 + 7.5 + 7.5 + 7.5 = 30 points]
\end{enumerate}

\item Suppose you are standing in a well-illuminated room with a large window, and you take a picture of the scene outside. The window undesirably acts as a semi-reflecting surface, and hence the picture will contain a reflection of the scene inside the room, besides the scene outside. While solutions exist for separating the two components from a single picture, here you will look at a simpler-to-solve version of this problem where you would take two pictures. The first picture $g_1$ is taken by adjusting your camera lens so that the scene outside ($f_1$) is in focus (we will assume that the scene outside has negligible depth variation when compared to the distance from the camera, and so it makes sense to say that the entire scene outside is in focus), and the reflection off the window surface ($f_2$) will now be defocussed or blurred.  This can be written as $g_1 = f_1 + h_2 * f_2$ where $h_2$ stands for the blur kernel that acted on $f_2$. The second picture $g_2$ is taken by focusing the camera onto the surface of the window, with the scene outside being defocussed. This can be written as $g_2 = h_1 * f_1 + f_2$ where $h_1$ is the blur kernel acting on $f_1$. Given $g_1$ and $g_2$, and assuming $h_1$ and $h_2$ are known, your task is to derive a formula to determine $f_1$ and $f_2$. Note that we are making the simplifying assumption that there was no relative motion between the camera and the scene outside while the two pictures were being acquired, and that there were no changes whatsoever to the scene outside or inside. Even with all these assumptions, you will notice something inherently problematic about the formula you will derive. What is it? \textsf{[5+5 = 10 points]}


\end{enumerate}
\end{document}

